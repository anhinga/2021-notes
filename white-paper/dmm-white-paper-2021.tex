\documentclass{article}

\usepackage{hyperref}

\baselineskip=15pt minus 1pt
\overfullrule=0pt

% maximize page usage
\oddsidemargin 1pt
\evensidemargin 1pt
\marginparwidth 30pt % these gain 53pt width
\topmargin 1pt       % gains 26pt height
\headheight 1pt      % gains 11pt height
\headsep 1pt         % gains 24pt height
%\footheight 12 pt % cannot be changed as number must fit
\footskip 24pt       % gains 6pt height
\textheight % 528 + 26 + 11 + 24 + 6 + 90 for luck
            685pt
\textwidth % 360 + 53 + 47 for luck
           480pt
% end of page layout changes

\begin{document}

\renewcommand{\abstractname}{\vspace{-\baselineskip}}

\renewcommand\contentsname{\vspace{-\baselineskip}}


\begin{center}

{\bf Dataflow Matrix Machines: a White Paper}
                                   



\vspace{0.1in}
Michael A. Bukatin


\vspace{0.085in}
January 15, 2021

\end{center}



\begin{abstract}

{\bf Dataflow matrix machines} form a {\bf novel class of
neural machines} with remarkable properties.
They combine {\bf general-purpose programming powers of stream-oriented
architectures} such as traditional dataflow programming and
more novel functional reactive programming with {\bf good machine learning
properties of conventional neural networks.}

\end{abstract}

\begin{center}
\line(1,0){250}
\end{center}

\section{Future potential}

The future potential and promise of dataflow matrix machines (DMMs) is high.

\begin{itemize}

  \item {\bf Program synthesis.} The task of synthesis of dataflow matrix machines
should be more tractable than conventional program synthesis. When one works with DMMs, the task of
{\em learning program sketches} is reformulated as {\em neural architecture search},
and converting a program sketch to a full program should be done by
conventional methods of neural net training. The dimension of the network and the dimension
of data are decoupled, so one has an option to synthesize compact neural machines
under this approach, if so desired.

\vspace{0.1in}
Dataflow matrix machines combine

  \begin{itemize}
      \item aspects of {\em program synthesis} setup\\ (compact, human-readable programs);
      \item aspects of {\em program inference} setup\\ (continuous models defined by matrices).
  \end{itemize}

\vspace{0.1in}
\item {\bf Self-modification and learning to learn.} Using neural networks for metalearning
is always non-trivial. In particular, dimension mismatch, namely the number of neuron outputs 
being much smaller than the number of network weights,
means that a neural network
can only modify itself in a highly constrained manner. Dataflow matrix machines address
this problem and have {\bf powerful and flexible self-modification facilities}.

\vspace{0.1in}
Therefore, a dataflow matrix machine can be equipped with a variety of primitives
performing self-modifications, and it can fruitfully learn various linear combinations and
compositions involving those primitives.

\vspace{0.1in}
Self-modification facilities of dataflow matrix machines are not limited to the weight
changes for the existing connections in the network. The available primitives allow to
modify the network topology as well. For example, primitives allowing the network
to control its own fractal-like growth by the means of cloning its own subnetworks
are available.

\vspace{0.1in}
So, this is a very promising architecture not only for methods of learning to learn
better in a traditional sense, but also for methods of learning to perform
neural architecture search better. 

\vspace{0.1in}
A dataflow matrix machine can comfortably host
an evolving population of other DMMs inside itself, so it is
an excellent environment for neuroevolution experiments and, in particular,
for the experiments aiming to learn to evolve better (or to evolve to evolve better).

\vspace{0.1in}
\item {\bf Neuro-symbolic architectures and hierarchical learning.} The availability of 
{\bf flexible tensors} based on {\em tree-shaped indices} makes this architecture
quite promising for neuro-symbolic methods and for hierarchies.

\vspace{0.1in}
\item {\bf Visual animation based on composition of unit generators.} DMMs 
generalize digital audio synthesis via composition of unit generators, but work
with more general streams than just streams of numbers. The old-fashioned
analog video synthesizers also work in the style of composition of unit
generators. Modern computer graphics tends to be imperative, oriented towards
specific data flows implemented inside GPUs, and to require the software practitioners to
focus on manual optimizations. The promise of doing animations via functional reactive
programming is to focus again on semantically meaningful
data flows, and to leave the hardware-oriented optimization to the underlying system.
Dataflow matrix machines aspire to make this promise a reality.


\end{itemize}

\section{How they work}

The essence of neural model of computations is that linear and non-linear computations are interleaved. Hence, the natural
degree of generality for neuromorphic computations is to work not with streams of numbers, but with arbitrary streams
supporting the notion of linear combination of several streams ({\bf linear streams}).


\vspace{0.1in}
\noindent
Dataflow matrix machines (DMMs) form a novel class of neural machines, which work with wide variety
of {\bf linear streams} instead of streams of numbers. The neurons have
arbitrary arity (arity of a neuron can be fixed or variable). Of particular note are
self-referential facilities: ability to change weights, topology, and the size of the active part of the network dynamically, on the fly,
and the reflection capability (the ability of the network to analyze its current configuration).

\vspace{0.1in}
\noindent
There are various kinds of linear streams. They include streams of numbers, sparse vectors and sparse tensors (both of
finite and infinite dimension), streams of functions and distributions. We found streams of V-values
({\bf flexible tensors} based on tree-shaped indices) to be of particular use.

\vspace{0.1in}
\noindent
A single dataflow matrix machine can process a large variety of different kinds of linear streams, or
it can be based on a single kind of linear streams, sufficiently expressive for a given class of situations.

\section{What have been done}

During the last seven years, I have been leading the efforts which resulted in
the discovery and study of dataflow matrix machines. The work was done by
several research collaborations between members of American and British 
academic communities. We published three papers and a number of preprints and
research notes, and released several related open source research-grade
implementations. During this period, I gave about 20 presentations on 
related topics at various venues in the United States, Canada, and Europe.

\vspace{0.1in}
\noindent
We investigated a variety of programming patterns and primitives in
dataflow matrix machines and performed experiments with various uses of self-referential
facilities.

\vspace{0.1in}
\noindent
We are continuing our efforts towards the next generation of DMMs 
and towards future programming and machine learning
technologies based on DMMs. We are also exploring 
deep connections between DMMs and attention-based models including Transformers.

\vspace{0.2in}
\noindent
{\bf Dataflow Matrix Machines links:}

\vspace{0.1in}
GitHub Pages: \href{https://anhinga.github.io/}{\tt https://anhinga.github.io}

\vspace{0.1in}
Open source implementation (Clojure): \href{https://github.com/jsa-aerial/DMM}{\tt https://github.com/jsa-aerial/DMM}

\vspace{0.1in}
One-page overview: \href{https://www.cs.brandeis.edu/~bukatin/dataflow-matrix-machines-2016.pdf}{\tt https://www.cs.brandeis.edu/$\sim$bukatin/dataflow-matrix-machines-2016.pdf}

\vspace{0.1in}
Reference slide deck: \href{https://researcher.watson.ibm.com/researcher/files/us-lmandel/aisys18-bukatin.pdf}{\footnotesize\tt https://researcher.watson.ibm.com/researcher/files/us-lmandel/aisys18-bukatin.pdf}

\vspace{0.1in}
Reference paper: \href{https://arxiv.org/abs/1712.07447}{\tt https://arxiv.org/abs/1712.07447}

\vspace{0.3in}
\noindent
{\bf Interdisciplinary collaborative research agenda (August 2020):}

\vspace{0.1in}
\href{https://www.cs.brandeis.edu/~bukatin/dmm-collaborative-research-agenda.pdf}{\tt https://www.cs.brandeis.edu/$\sim$bukatin/dmm-collaborative-research-agenda.pdf}







\end{document}